% Chapter 1
% 
\chapter{Introduction} % Main chapter title
\label{chap:Chapter1} % For referencing the chapter elsewhere, use Chapter~\ref{Chapter1}


%-------------------------------------------------------------------------------
%---------
%

This chapter introduces the dissertation developed within the Master's program in Software Engineering at the Instituto Politécnico do Porto (ISEP). It begins by presenting the context and defining the problem that serves as the basis for the literature review. The following section specifies the objectives and methodology applied during the development process. In addition, the chapter includes a critical analysis section addressing the technical implications of the developed solution, as well as a section on the use of Artificial Intelligence (AI) during the writing of the thesis. The final section presents the overall organization of the document.

\section{Context}
\label{sec:chap1_context}

With the increasing adoption of AI in professional contexts, the integration of Machine Learning (ML) has become widespread. To ensure accurate predictions, ML models require training on relevant datasets. However, many real-world datasets are inherently imbalanced, in which one or more classes are significantly underrepresented, a condition known as class imbalance (CI). Although this issue is well addressed for binary classification through various techniques and tools, ML with imbalanced multiclass data remains a complex challenge (\cite{chen2024survey}).

Class imbalance  poses a significant challenge in machine learning classification tasks, as minority classes often contain an insufficient number of samples for the model to adequately learn their patterns. This shortage of representative data limits the model’s ability to generalize and leads to biased predictions, where the classifier tends to misclassify rare instances or does not recognize these instances as unique classes, instead aggregating them with majority classes (\cite{olamendy2024practical}). The resulting accuracy degradation is particularly problematic in applications where minority classes are of critical importance, such as fraud detection, medical diagnosis, bio-informatics, or telecommunications. In these contexts, overlooking infrequent but meaningful events can lead to severe consequences, highlighting the necessity of developing methods that effectively address the imbalance and ensure reliable model performance across all classes.


\section{Problem}
\label{sec:chap1_problem}


Despite the development of numerous algorithms designed to address multiclass imbalance and classifications, practical challenges remain in applying these solutions effectively. Real-world datasets often present complex distributions, and selecting an appropriate balancing method requires careful consideration of the dataset’s characteristics. As Vieira (2024) notes, "The best approach to handle imbalanced data highly depends on the nature of the data. The methods and combination of methods proposed are abundant in various conceivable outcomes, and most times they require specialised knowledge to be used correctly" (\cite{Vieira2024}). This observation highlights the difficulty in utilizing the correct balancing methods for a given dataset, where  optimal utilization of balancing methods often proves to be time-intensive. In practical terms, this can result in trade-offs during the training of a machine learning model, where the selection of methods may negatively impact performance, reduce predictive accuracy, or extend development time.

Existing software tools often focus on binary classification problems or offer only a limited set of balancing methods, providing little support for multiclass scenarios. Consequently, there is a clear need for a tool that not only can implement multiple balancing algorithms but can also recommend the most appropriate approach based on the specific characteristics of the dataset.

\section{Objective}
\label{sec:chap1_objective}

This project seeks to extend and improve an existing open-source tool (\cite{vieira2025github}) by enabling support for multiclass imbalanced learning. The enhancement focuses on integrating a comprehensive set of state-of-the-art balancing algorithms and implementing intelligent recommendation mechanisms that consider the specific characteristics of the dataset. The goal is to create a robust software tool capable of addressing the challenges posed by imbalanced multiclass datasets while reducing reliance on manual selection of techniques.
The final tool will implement a variety of multiclass balancing strategies, including both data-level and algorithm-level approaches, and assess their impact across multiple classifiers to identify performance variations. In addition, it will generate recommendations for the most appropriate balancing method based on empirical evaluation and meta-learning logic, thereby supporting informed decision-making and improving overall model performance. By combining algorithm integration, empirical assessment, and automated guidance, the tool aims to assist researchers and practitioners in efficiently managing imbalanced multiclass datasets while minimizing performance trade-offs and development overhead.

\section{Methodology}
\label{sec:chap1_methodology}

The work conducted in this dissertation can be divided into two main components: the research component, which primarily supports the literature review presented in Chapter 2, and the development component, which encompasses the implementation of the proposed solution detailed in Chapters 3 and 4. Each component followed a specific methodology appropriate to its objectives and scope. The following sections describe these methodologies in detail, highlighting the procedures, tools, and reasoning that guided both the theoretical research and the practical development phases of the project.

\subsection{Research Methodology}
\label{subsec:chap1_research_methodology}

The literature reviewed in this dissertation focuses on the stage of the machine learning dedicated to model training through dataset processing. Specifically, it examines this stage in the context of real-world scenarios where datasets are imbalanced, containing multiple classes with significantly fewer data points compared to dominant majority classes. The objective is to contextualize the effects of class imbalance on model training, exploring methods for re-balancing data, classification, and different metrics to analyze the performance and accuracy of the training. This investigation touched on several key topics, including:

%list of items tackled:
\begin{itemize}
    \item \textbf{Imbalanced datasets}: This section presents an initial examination of class imbalance and its impact on the performance of machine learning models. It compares the accuracy of models trained with balanced and imbalanced datasets to demonstrate the degradation caused by unequal class distributions. In addition, it reviews multiple studies across various professional domains, highlighting different strategies that have been adopted to mitigate the effects of imbalance learning.

    \item \textbf{Balancing methods}: This section explores the two main categories of data balancing techniques, under-sampling and oversampling, while examining their most common implementations. It provides an overview of several commonly used methods, including \textit{SMOTE} variants, \textit{MDO}, \textit{MLSMOTE}, and \textit{ADASYN}, among others, detailing how each approach achieves improved class representation.

    \item \textbf{Classification algorithms}: This section discusses several classification algorithms applied to imbalanced datasets, identifying their advantages and potential drawbacks, depending on the characteristics of the dataset. Algorithms such as \textit{Random Forest}, \textit{AdaBoost}, \textit{XGBoost}, \textit{LightGBM}, \textit{MLP}, and \textit{One-versus-Rest} variants are examined in terms of their suitability for handling multi-class imbalance and their comparative performance under different conditions.

    \item \textbf{Meta-learning}: A dedicated section investigates meta-learning approaches relevant to the development of the recommendation module in the proposed solution, focusing on how meta-learning techniques can support the automated selection of balancing and classification strategies based on dataset properties and prior experimental outcomes.

    \item \textbf{Evaluation metrics}: This section analyzes multiple evaluation metrics to determine the most appropriate combination of metrics for reliable model assessment in the context of model training with imbalanced datasets, where a single metric is insufficient to accurately assess performance.

\end{itemize}

All referenced and reviewed papers were obtained from popular academic databases, including \textit{ResearchGate}, \textit{Springer}, \textit{ScienceDirect} and \textit{Google Scholar}. The selection prioritized peer-reviewed publications released within the past five years that demonstrated academic relevance and impact, with citation numbers used as the main metric to determine relevance. Targeted search queries were formulated to limit the scope of the  results, with key phrases such as "Imbalanced dataset handling" and "classification problems with imbalanced learning." being used.

\subsection{Development Methodology}
\label{sebsec:chap1_development_methodology}
The methodology selected for the development section of the dissertation is \textit{Cross Industry Standard Process for Data Mining} (CRISP-DM), a commonly used methodology for data mining and data science projects that proposes the use of multiple sequential phases to guide the project from initial understanding to deployment. This methodology was selected due to its ease of implementation, adaptability to the project context, and its established use in both academic and professional settings (\cite{DataSciencePM_CrispDM2}). 

This section provides a detailed description of each CRISP-DM phase (\cite{Shearer2000}) and illustrates how these phases are applied within the context of the proposed solution, each phase and how they connect to each other can be seen in Figure \ref{fig:crispdm_diagram}.

\begin{figure}[h!]
\centering
\includegraphics[scale=0.5]{ch1/assets/CRISPDM}
\caption[CRSIP-DM Process Diagram]{CRSIP-DM Process Diagram.}
\label{fig:crispdm_diagram}
\end{figure}

\begin{itemize}
    \item \textbf{Business Understanding:} This phase focuses on defining the project objectives and requirements from a practical perspective. It identifies the problem to be addressed, and converts this knowledge into a data mining problem, allowing a definition of necessary goals that need to be achieved. In the context of this dissertation, this involves understanding the limitations of existing solutions for imbalanced multiclass classification and defining the scope for developing the tool.

    \item \textbf{Data Understanding:} This phase involves the initial collection and examination of datasets to gain insight into their structure, quality, and potential issues, in this phase, all data collected should go through a verification process to guarantee the data's quality. For this project, this phase centers on finding suitable datasets, that are in multiple levels of imbalance, with different numbers of classes, data amount and parameters specifications, which will help cover a comprehensive number of scenarios that will be used for training the recommendation engine.

    \item \textbf{Data Preparation:} This phase encompasses all tasks required to prepare the datasets used for modeling, such as cleaning, transforming, and organizing the data. In the context of this solution, this phase focuses on integrating multiple balancing methods and classification algorithms, and processing all extracted datasets through all possible combinations. the resulting metrics will be combined with the dataset's characteristics to fill the knowledge base.


    \item \textbf{Modeling:} This phase focuses on developing and training the recommendation engine that forms the core of the proposed solution. The engine is trained using the knowledge base obtained in the data preparation. Through this process, relationships between dataset parameters and optimal method combinations are learned, enabling the system to recommend suitable balancing strategies and classifiers for new datasets. This phase also involves fine-tuning the learning logic to improve the accuracy and reliability of the recommendations.


    \item \textbf{Evaluation:} This phase focuses on evaluating the performance of the trained recommendation engine against the defined metrics, to see if the generated recommendations align with the expected outcomes and whether the system can consistently identify effective combinations of balancing methods and classifiers.

    \item \textbf{Deployment:} This phase typically represents the stage where a finalized version of the developed solution is deployed or made available for use. In the context of this solution, however, it will serve as a stage for consolidating, documenting, and analyzing the outcomes of the previous phases. The development chapters of this dissertation will act as the medium for reporting all findings, analyses, and insights obtained from this phase.

\end{itemize}


Alongside CRISP-DM, \textit{Scrum} methodology will be implemented as an agile approach to project management, with monthly sprints being used to organize and plan the work tasks. At the conclusion of each sprint, a meeting with the dissertation supervisor will be conducted to review progress, gather feedback, and adjust the objectives and priorities for the following sprint as required.


\section{Structure of the document}
\label{sec:chap1_structure_of_the_document}

The dissertation is organized into five chapters, each addressing a specific component of the research and development process. The first chapter introduces the overall context and the central problem explored in the study. It defines the primary development objective, outlines the methodology adopted for both research and implementation, and includes a critical analysis of the necessity and potential impact of the proposed tool. This chapter also establishes the foundation upon which subsequent chapters are built, ensuring a coherent transition from problem identification to solution development.

The second chapter presents a comprehensive literature review covering the fundamental topics relevant to the dissertation. It begins by discussing datasets and the distinctions between binary and multiclass configurations, emphasizing how data distribution impacts model learning. The chapter proceeds to examine hyperparameter optimization, describing its role in improving model performance and the techniques used for its adjustment. A detailed analysis of data balancing follows, explaining oversampling and under-sampling methods, along with the most widely used techniques in each category and their practical implications. Further sections explore classification algorithms, highlighting their behavior under imbalanced conditions and comparing their suitability for different types of datasets. Finally, a section to explore potential evaluation metrics, addressing the importance of using multiple measures to ensure a reliable and comprehensive assessment of model performance.

The third chapter focuses on the practical development of the proposed tool. It describes the selection and preparation of datasets, the implementation of balancing methods, and the integration of classification algorithms. This chapter also includes an analysis of the interaction between balancing methods and classifiers, with the aim of identifying optimal combinations for datasets with distinct characteristics. The development of the learning module, which incorporates these insights into the tool’s recommendation logic, is also detailed. This chapter serves as the technical core of the dissertation, linking theoretical knowledge with applied software development.

The fourth chapter presents the evaluation and analysis of the results. It outlines the performance metrics adopted for validation, including those used to assess time efficiency and accuracy. Comparative analyses are conducted between the developed tool and existing solutions, demonstrating the tool’s ability to improve classification outcomes while maintaining accuracy. The discussion also reflects on the trade-offs observed during experimentation and interprets the significance of the obtained results in relation to the initial objectives.

The final chapter provides the conclusion of the dissertation, summarizing the findings and assessing whether the enhanced tool delivers measurable improvements over existing approaches. It also reflects on the broader implications of the results, identifies limitations encountered during the study, and suggests potential directions for future research and tool refinement.

\section{Use of AI-generated content}
\label{sec:chap1_use_of_ai_generated_content}

To ensure compliance with the guidelines established by PREPD and DIMEI regarding the use of AI-generated content, this section outlines the use of AI during the development of the project and its specific purposes. The use of AI was restricted to grammatical and orthographic revision during the preparation of this document. ChatGPT was employed for proofreading selected text segments to enhance clarity and maintain consistency and structured writing.

The goal was to achieve writing that was free of mistakes and followed good conventions for academic writing. As such, for each instance of proofreading, the prompt was preceded by a defined set of rules intended to restrict the tool’s scope of assistance and ensure controlled outcomes. These criteria, established by the author, are as follow: use of assertive language, simple vocabulary, active voice, while avoiding adjectives, synonyms, imprecise expressions, and repetition. 

All AI-assisted outputs were manually reviewed by the author to ensure that the original meaning was preserved and that no factual errors or hallucinated content were introduced. It is important to emphasize that all ideas, analyses, methodologies, development processes, and interpretation of results presented in this dissertation were entirely conceived and executed by the author, independent of AI assistance.
