% Chapter 1
% 
\chapter{Introduction} % Main chapter title
\label{chap:Chapter1} % For referencing the chapter elsewhere, use Chapter~\ref{Chapter1}


%-------------------------------------------------------------------------------
%---------
%

This chapter introduces the dissertation developed within the Master's program in Software Engineering at the Instituto Politécnico do Porto (ISEP). It begins by presenting the context and defining the problem that serves as the basis for the literature review. The following section specifies the objectives and methodology applied during the development process. In addition, the chapter includes a critical analysis section addressing the technical implications of the developed solution, as well as a section on the use of Artificial Intelligence (AI) during the writing of the thesis. The final section presents the overall organization of the document.

\section{Context}
\label{sec:chap1_context}

With the increasing adoption of AI in professional contexts, the integration of Machine Learning (ML) has become widespread. To ensure accurate predictions, ML models require training on relevant datasets. However, many real-world datasets are inherently imbalanced, in which one or more classes are significantly underrepresented, a condition known as class imbalance (CI). Although this issue is well addressed for binary classification through various techniques and tools, ML with imbalanced multiclass data remains a complex challenge (\cite{chen2024survey}).

Class imbalance  poses a significant challenge in machine learning classification tasks, as minority classes often contain an insufficient number of samples for the model to adequately learn their patterns. This shortage of representative data limits the model’s ability to generalize and leads to biased predictions, where the classifier tends to misclassify rare instances or does not recognize these instances as unique classes, instead aggregating them with majority classes (\cite{olamendy2024practical}). The resulting accuracy degradation is particularly problematic in applications where minority classes are of critical importance, such as fraud detection, medical diagnosis, bio-informatics, or telecommunications. In these contexts, overlooking infrequent but meaningful events can lead to severe consequences, highlighting the necessity of developing methods that effectively address the imbalance and ensure reliable model performance across all classes.


\section{Problem}
\label{sec:chap1_problem}


Despite the development of numerous algorithms designed to address multiclass imbalance and classifications, practical challenges remain in applying these solutions effectively. Real-world datasets often present complex distributions, and selecting an appropriate balancing method requires careful consideration of the dataset’s characteristics. As Vieira (2024) notes, "The best approach to handle imbalanced data highly depends on the nature of the data. The methods and combination of methods proposed are abundant in various conceivable outcomes, and most times they require specialised knowledge to be used correctly" (\cite{Vieira2024}). This observation highlights the difficulty in utilizing the correct balancing methods for a given dataset, where  optimal utilization of balancing methods often proves to be time-intensive. In practical terms, this can result in trade-offs during the training of a machine learning model, where the selection of methods may negatively impact performance, reduce predictive accuracy, or extend development time.

Existing software tools often focus on binary classification problems or offer only a limited set of balancing methods, providing little support for multiclass scenarios. Consequently, there is a clear need for a tool that not only can implement multiple balancing algorithms but can also recommend the most appropriate approach based on the specific characteristics of the dataset.

\section{Objective}
\label{sec:chap1_objective}

This project seeks to extend and improve an existing open-source tool (\cite{vieira2025github}) by enabling support for multiclass imbalanced learning. The enhancement focuses on integrating a comprehensive set of state-of-the-art balancing algorithms and implementing intelligent recommendation mechanisms that consider the specific characteristics of the dataset. The goal is to create a robust software tool capable of addressing the challenges posed by imbalanced multiclass datasets while reducing reliance on manual selection of techniques.
The final tool will implement a variety of multiclass balancing strategies, including both data-level and algorithm-level approaches, and assess their impact across multiple classifiers to identify performance variations. In addition, it will generate recommendations for the most appropriate balancing method based on empirical evaluation and meta-learning logic, thereby supporting informed decision-making and improving overall model performance. By combining algorithm integration, empirical assessment, and automated guidance, the tool aims to assist researchers and practitioners in efficiently managing imbalanced multiclass datasets while minimizing performance trade-offs and development overhead.

\section{Methodology}
\label{sec:chap1_methodology}

\section{Critical Analysis}
\label{sec:chap1_analytical_critical_and_ethical_analysis}

\section{Structure of the document}
\label{sec:chap1_structure_of_the_document}

\section{Use of AI-generated content}
\label{sec:chap1_use_of_ai_generated_content}
