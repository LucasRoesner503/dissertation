% Chapter 1
% 
\chapter{Introduction} % Main chapter title
\label{chap:Chapter1} % For referencing the chapter elsewhere, use Chapter~\ref{Chapter1}


%-------------------------------------------------------------------------------
%---------
%

This chapter introduces the dissertation developed within the Master's program in Software Engineering at the Instituto Politécnico do Porto (ISEP). It begins by presenting the context and defining the problem that serves as the basis for the literature review. The following section specifies the objectives and methodology applied during the development process. In addition, the chapter includes a critical analysis section addressing the technical implications of the developed solution, as well as a section on the use of Artificial Intelligence (AI) during the writing of the thesis. The final section presents the overall organization of the document.

\section{Context}
\label{sec:chap1_context}

With the ever expanding adoption of AI in professional contexts, many Large Learning Models (LLMs) are trained with imbalanced datasets, in which one or more classes are significantly underrepresented, a condition known as class imbalance.(ref) Although this issue is well addressed for binary classification through various techniques and tools, machine learning with imbalanced multiclass data remains a complex challenge. (ref) 

Class imbalance poses a significant challenge in machine learning classification tasks, as minority classes often contain an insufficient number of samples for the model to adequately learn their patterns. This shortage of representative data limits the model’s ability to generalize and leads to biased predictions, where the classifier tends to misclassify rare instances or does not recognize these instances as unique classes (ref). The resulting performance degradation is particularly problematic in applications where minority classes are of critical importance, such as fraud detection, medical diagnosis, bio-informatics, or telecommunications. In these contexts, overlooking infrequent but meaningful events can lead to severe consequences, highlighting the necessity of developing methods that effectively address the imbalance and ensure reliable model performance across all classes.


\section{Problem}
\label{sec:chap1_problem}

Despite the existence of several algorithms proposed to handle multiclass imbalance, there is still a lack of software tools that integrate these approaches and recommend the most suitable technique based on dataset characteristics. Most existing solutions support only binary problems or provide a limited range of balancing strategies without guidance on how to select the most appropriate option.

\section{Objective}
\label{sec:chap1_objective}

\section{Methodology}
\label{sec:chap1_methodology}

\section{Critical Analysis}
\label{sec:chap1_analytical_critical_and_ethical_analysis}

\section{Structure of the document}
\label{sec:chap1_structure_of_the_document}

\section{Use of AI-generated content}
\label{sec:chap1_use_of_ai_generated_content}
